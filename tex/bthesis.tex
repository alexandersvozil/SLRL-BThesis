% Author: Alexander Svozil
% Matr.-nr.: 1026213
% TU Wien
\documentclass [12pt]{article}
\usepackage [utf8]{inputenc}


\begin{document}
\author{Alexander Svozil}
\title{Bachelor Thesis \\ The Server Location Problem with Restricted Loads 
on Servers and Links}

\maketitle

\section{Abstract}
The server location problem with restricted loads on servers and links (SLRL) is an NP-Complete
problem, introduced by Hiroyoshi Miwa et al. in the Paper "Method of Locating Mirror Servers
to Alleviate Load on Servers and Links" \cite{mirrorserver}. The problem 
came up, because the massive volume of data distributed by content delivery networks (CDNs) 
require well located mirror servers in order not to badly influence the quality of their service.
Two examples for CDNs would be "Amazon CloudFront" a traditional commercial CDN, or the "AT\&T Inc."
a Telco CDN which has advantages over traditional CDNs because they own the so called "last mile",
the final leg of the telecommunications networks. CDN nodes are usually deployed in multiple 
locations, often over multiple backbones, reaching thousands of nodes with tens of thousands of 
servers. \cite{wiki:cdn} \cite{wiki:lastmile}
The two constraints induced by the choice of mirror servers is the number of maximum nodes
accessing a mirror server and maximum number of nodes accessing a mirror server 
through a specific link. The constraint mentioned first, corresponds to the network load on
the (mirror-)servers. The second constraint corresponds to the restricted load on the links.
First I want to prove that SLRL is NP-complete. Next, I want to propose my own algorithm as a
sequel to the existing greedy algorithm proposed by Miwa et al. \cite{mirrorserver}.


\tableofcontents
\section{Introduction}
\section{Implementation}
\section{Conclusion}

\bibliography{bibtex}
\bibliographystyle{alpha}
\end{document}
